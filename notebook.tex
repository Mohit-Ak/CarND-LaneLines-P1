
% Default to the notebook output style

    


% Inherit from the specified cell style.




    
\documentclass[11pt]{article}

    
    
    \usepackage[T1]{fontenc}
    % Nicer default font (+ math font) than Computer Modern for most use cases
    \usepackage{mathpazo}

    % Basic figure setup, for now with no caption control since it's done
    % automatically by Pandoc (which extracts ![](path) syntax from Markdown).
    \usepackage{graphicx}
    % We will generate all images so they have a width \maxwidth. This means
    % that they will get their normal width if they fit onto the page, but
    % are scaled down if they would overflow the margins.
    \makeatletter
    \def\maxwidth{\ifdim\Gin@nat@width>\linewidth\linewidth
    \else\Gin@nat@width\fi}
    \makeatother
    \let\Oldincludegraphics\includegraphics
    % Set max figure width to be 80% of text width, for now hardcoded.
    \renewcommand{\includegraphics}[1]{\Oldincludegraphics[width=.8\maxwidth]{#1}}
    % Ensure that by default, figures have no caption (until we provide a
    % proper Figure object with a Caption API and a way to capture that
    % in the conversion process - todo).
    \usepackage{caption}
    \DeclareCaptionLabelFormat{nolabel}{}
    \captionsetup{labelformat=nolabel}

    \usepackage{adjustbox} % Used to constrain images to a maximum size 
    \usepackage{xcolor} % Allow colors to be defined
    \usepackage{enumerate} % Needed for markdown enumerations to work
    \usepackage{geometry} % Used to adjust the document margins
    \usepackage{amsmath} % Equations
    \usepackage{amssymb} % Equations
    \usepackage{textcomp} % defines textquotesingle
    % Hack from http://tex.stackexchange.com/a/47451/13684:
    \AtBeginDocument{%
        \def\PYZsq{\textquotesingle}% Upright quotes in Pygmentized code
    }
    \usepackage{upquote} % Upright quotes for verbatim code
    \usepackage{eurosym} % defines \euro
    \usepackage[mathletters]{ucs} % Extended unicode (utf-8) support
    \usepackage[utf8x]{inputenc} % Allow utf-8 characters in the tex document
    \usepackage{fancyvrb} % verbatim replacement that allows latex
    \usepackage{grffile} % extends the file name processing of package graphics 
                         % to support a larger range 
    % The hyperref package gives us a pdf with properly built
    % internal navigation ('pdf bookmarks' for the table of contents,
    % internal cross-reference links, web links for URLs, etc.)
    \usepackage{hyperref}
    \usepackage{longtable} % longtable support required by pandoc >1.10
    \usepackage{booktabs}  % table support for pandoc > 1.12.2
    \usepackage[inline]{enumitem} % IRkernel/repr support (it uses the enumerate* environment)
    \usepackage[normalem]{ulem} % ulem is needed to support strikethroughs (\sout)
                                % normalem makes italics be italics, not underlines
    

    
    
    % Colors for the hyperref package
    \definecolor{urlcolor}{rgb}{0,.145,.698}
    \definecolor{linkcolor}{rgb}{.71,0.21,0.01}
    \definecolor{citecolor}{rgb}{.12,.54,.11}

    % ANSI colors
    \definecolor{ansi-black}{HTML}{3E424D}
    \definecolor{ansi-black-intense}{HTML}{282C36}
    \definecolor{ansi-red}{HTML}{E75C58}
    \definecolor{ansi-red-intense}{HTML}{B22B31}
    \definecolor{ansi-green}{HTML}{00A250}
    \definecolor{ansi-green-intense}{HTML}{007427}
    \definecolor{ansi-yellow}{HTML}{DDB62B}
    \definecolor{ansi-yellow-intense}{HTML}{B27D12}
    \definecolor{ansi-blue}{HTML}{208FFB}
    \definecolor{ansi-blue-intense}{HTML}{0065CA}
    \definecolor{ansi-magenta}{HTML}{D160C4}
    \definecolor{ansi-magenta-intense}{HTML}{A03196}
    \definecolor{ansi-cyan}{HTML}{60C6C8}
    \definecolor{ansi-cyan-intense}{HTML}{258F8F}
    \definecolor{ansi-white}{HTML}{C5C1B4}
    \definecolor{ansi-white-intense}{HTML}{A1A6B2}

    % commands and environments needed by pandoc snippets
    % extracted from the output of `pandoc -s`
    \providecommand{\tightlist}{%
      \setlength{\itemsep}{0pt}\setlength{\parskip}{0pt}}
    \DefineVerbatimEnvironment{Highlighting}{Verbatim}{commandchars=\\\{\}}
    % Add ',fontsize=\small' for more characters per line
    \newenvironment{Shaded}{}{}
    \newcommand{\KeywordTok}[1]{\textcolor[rgb]{0.00,0.44,0.13}{\textbf{{#1}}}}
    \newcommand{\DataTypeTok}[1]{\textcolor[rgb]{0.56,0.13,0.00}{{#1}}}
    \newcommand{\DecValTok}[1]{\textcolor[rgb]{0.25,0.63,0.44}{{#1}}}
    \newcommand{\BaseNTok}[1]{\textcolor[rgb]{0.25,0.63,0.44}{{#1}}}
    \newcommand{\FloatTok}[1]{\textcolor[rgb]{0.25,0.63,0.44}{{#1}}}
    \newcommand{\CharTok}[1]{\textcolor[rgb]{0.25,0.44,0.63}{{#1}}}
    \newcommand{\StringTok}[1]{\textcolor[rgb]{0.25,0.44,0.63}{{#1}}}
    \newcommand{\CommentTok}[1]{\textcolor[rgb]{0.38,0.63,0.69}{\textit{{#1}}}}
    \newcommand{\OtherTok}[1]{\textcolor[rgb]{0.00,0.44,0.13}{{#1}}}
    \newcommand{\AlertTok}[1]{\textcolor[rgb]{1.00,0.00,0.00}{\textbf{{#1}}}}
    \newcommand{\FunctionTok}[1]{\textcolor[rgb]{0.02,0.16,0.49}{{#1}}}
    \newcommand{\RegionMarkerTok}[1]{{#1}}
    \newcommand{\ErrorTok}[1]{\textcolor[rgb]{1.00,0.00,0.00}{\textbf{{#1}}}}
    \newcommand{\NormalTok}[1]{{#1}}
    
    % Additional commands for more recent versions of Pandoc
    \newcommand{\ConstantTok}[1]{\textcolor[rgb]{0.53,0.00,0.00}{{#1}}}
    \newcommand{\SpecialCharTok}[1]{\textcolor[rgb]{0.25,0.44,0.63}{{#1}}}
    \newcommand{\VerbatimStringTok}[1]{\textcolor[rgb]{0.25,0.44,0.63}{{#1}}}
    \newcommand{\SpecialStringTok}[1]{\textcolor[rgb]{0.73,0.40,0.53}{{#1}}}
    \newcommand{\ImportTok}[1]{{#1}}
    \newcommand{\DocumentationTok}[1]{\textcolor[rgb]{0.73,0.13,0.13}{\textit{{#1}}}}
    \newcommand{\AnnotationTok}[1]{\textcolor[rgb]{0.38,0.63,0.69}{\textbf{\textit{{#1}}}}}
    \newcommand{\CommentVarTok}[1]{\textcolor[rgb]{0.38,0.63,0.69}{\textbf{\textit{{#1}}}}}
    \newcommand{\VariableTok}[1]{\textcolor[rgb]{0.10,0.09,0.49}{{#1}}}
    \newcommand{\ControlFlowTok}[1]{\textcolor[rgb]{0.00,0.44,0.13}{\textbf{{#1}}}}
    \newcommand{\OperatorTok}[1]{\textcolor[rgb]{0.40,0.40,0.40}{{#1}}}
    \newcommand{\BuiltInTok}[1]{{#1}}
    \newcommand{\ExtensionTok}[1]{{#1}}
    \newcommand{\PreprocessorTok}[1]{\textcolor[rgb]{0.74,0.48,0.00}{{#1}}}
    \newcommand{\AttributeTok}[1]{\textcolor[rgb]{0.49,0.56,0.16}{{#1}}}
    \newcommand{\InformationTok}[1]{\textcolor[rgb]{0.38,0.63,0.69}{\textbf{\textit{{#1}}}}}
    \newcommand{\WarningTok}[1]{\textcolor[rgb]{0.38,0.63,0.69}{\textbf{\textit{{#1}}}}}
    
    
    % Define a nice break command that doesn't care if a line doesn't already
    % exist.
    \def\br{\hspace*{\fill} \\* }
    % Math Jax compatability definitions
    \def\gt{>}
    \def\lt{<}
    % Document parameters
    \title{Lane\_Detection\_Project}
    
    
    

    % Pygments definitions
    
\makeatletter
\def\PY@reset{\let\PY@it=\relax \let\PY@bf=\relax%
    \let\PY@ul=\relax \let\PY@tc=\relax%
    \let\PY@bc=\relax \let\PY@ff=\relax}
\def\PY@tok#1{\csname PY@tok@#1\endcsname}
\def\PY@toks#1+{\ifx\relax#1\empty\else%
    \PY@tok{#1}\expandafter\PY@toks\fi}
\def\PY@do#1{\PY@bc{\PY@tc{\PY@ul{%
    \PY@it{\PY@bf{\PY@ff{#1}}}}}}}
\def\PY#1#2{\PY@reset\PY@toks#1+\relax+\PY@do{#2}}

\expandafter\def\csname PY@tok@kd\endcsname{\let\PY@bf=\textbf\def\PY@tc##1{\textcolor[rgb]{0.00,0.50,0.00}{##1}}}
\expandafter\def\csname PY@tok@k\endcsname{\let\PY@bf=\textbf\def\PY@tc##1{\textcolor[rgb]{0.00,0.50,0.00}{##1}}}
\expandafter\def\csname PY@tok@sr\endcsname{\def\PY@tc##1{\textcolor[rgb]{0.73,0.40,0.53}{##1}}}
\expandafter\def\csname PY@tok@vg\endcsname{\def\PY@tc##1{\textcolor[rgb]{0.10,0.09,0.49}{##1}}}
\expandafter\def\csname PY@tok@mi\endcsname{\def\PY@tc##1{\textcolor[rgb]{0.40,0.40,0.40}{##1}}}
\expandafter\def\csname PY@tok@mh\endcsname{\def\PY@tc##1{\textcolor[rgb]{0.40,0.40,0.40}{##1}}}
\expandafter\def\csname PY@tok@na\endcsname{\def\PY@tc##1{\textcolor[rgb]{0.49,0.56,0.16}{##1}}}
\expandafter\def\csname PY@tok@fm\endcsname{\def\PY@tc##1{\textcolor[rgb]{0.00,0.00,1.00}{##1}}}
\expandafter\def\csname PY@tok@ne\endcsname{\let\PY@bf=\textbf\def\PY@tc##1{\textcolor[rgb]{0.82,0.25,0.23}{##1}}}
\expandafter\def\csname PY@tok@nl\endcsname{\def\PY@tc##1{\textcolor[rgb]{0.63,0.63,0.00}{##1}}}
\expandafter\def\csname PY@tok@no\endcsname{\def\PY@tc##1{\textcolor[rgb]{0.53,0.00,0.00}{##1}}}
\expandafter\def\csname PY@tok@gi\endcsname{\def\PY@tc##1{\textcolor[rgb]{0.00,0.63,0.00}{##1}}}
\expandafter\def\csname PY@tok@nc\endcsname{\let\PY@bf=\textbf\def\PY@tc##1{\textcolor[rgb]{0.00,0.00,1.00}{##1}}}
\expandafter\def\csname PY@tok@m\endcsname{\def\PY@tc##1{\textcolor[rgb]{0.40,0.40,0.40}{##1}}}
\expandafter\def\csname PY@tok@se\endcsname{\let\PY@bf=\textbf\def\PY@tc##1{\textcolor[rgb]{0.73,0.40,0.13}{##1}}}
\expandafter\def\csname PY@tok@kc\endcsname{\let\PY@bf=\textbf\def\PY@tc##1{\textcolor[rgb]{0.00,0.50,0.00}{##1}}}
\expandafter\def\csname PY@tok@sh\endcsname{\def\PY@tc##1{\textcolor[rgb]{0.73,0.13,0.13}{##1}}}
\expandafter\def\csname PY@tok@ge\endcsname{\let\PY@it=\textit}
\expandafter\def\csname PY@tok@cp\endcsname{\def\PY@tc##1{\textcolor[rgb]{0.74,0.48,0.00}{##1}}}
\expandafter\def\csname PY@tok@cpf\endcsname{\let\PY@it=\textit\def\PY@tc##1{\textcolor[rgb]{0.25,0.50,0.50}{##1}}}
\expandafter\def\csname PY@tok@nb\endcsname{\def\PY@tc##1{\textcolor[rgb]{0.00,0.50,0.00}{##1}}}
\expandafter\def\csname PY@tok@gd\endcsname{\def\PY@tc##1{\textcolor[rgb]{0.63,0.00,0.00}{##1}}}
\expandafter\def\csname PY@tok@kr\endcsname{\let\PY@bf=\textbf\def\PY@tc##1{\textcolor[rgb]{0.00,0.50,0.00}{##1}}}
\expandafter\def\csname PY@tok@nt\endcsname{\let\PY@bf=\textbf\def\PY@tc##1{\textcolor[rgb]{0.00,0.50,0.00}{##1}}}
\expandafter\def\csname PY@tok@c\endcsname{\let\PY@it=\textit\def\PY@tc##1{\textcolor[rgb]{0.25,0.50,0.50}{##1}}}
\expandafter\def\csname PY@tok@ss\endcsname{\def\PY@tc##1{\textcolor[rgb]{0.10,0.09,0.49}{##1}}}
\expandafter\def\csname PY@tok@w\endcsname{\def\PY@tc##1{\textcolor[rgb]{0.73,0.73,0.73}{##1}}}
\expandafter\def\csname PY@tok@nn\endcsname{\let\PY@bf=\textbf\def\PY@tc##1{\textcolor[rgb]{0.00,0.00,1.00}{##1}}}
\expandafter\def\csname PY@tok@cs\endcsname{\let\PY@it=\textit\def\PY@tc##1{\textcolor[rgb]{0.25,0.50,0.50}{##1}}}
\expandafter\def\csname PY@tok@gr\endcsname{\def\PY@tc##1{\textcolor[rgb]{1.00,0.00,0.00}{##1}}}
\expandafter\def\csname PY@tok@gh\endcsname{\let\PY@bf=\textbf\def\PY@tc##1{\textcolor[rgb]{0.00,0.00,0.50}{##1}}}
\expandafter\def\csname PY@tok@ow\endcsname{\let\PY@bf=\textbf\def\PY@tc##1{\textcolor[rgb]{0.67,0.13,1.00}{##1}}}
\expandafter\def\csname PY@tok@sa\endcsname{\def\PY@tc##1{\textcolor[rgb]{0.73,0.13,0.13}{##1}}}
\expandafter\def\csname PY@tok@kn\endcsname{\let\PY@bf=\textbf\def\PY@tc##1{\textcolor[rgb]{0.00,0.50,0.00}{##1}}}
\expandafter\def\csname PY@tok@o\endcsname{\def\PY@tc##1{\textcolor[rgb]{0.40,0.40,0.40}{##1}}}
\expandafter\def\csname PY@tok@ni\endcsname{\let\PY@bf=\textbf\def\PY@tc##1{\textcolor[rgb]{0.60,0.60,0.60}{##1}}}
\expandafter\def\csname PY@tok@sx\endcsname{\def\PY@tc##1{\textcolor[rgb]{0.00,0.50,0.00}{##1}}}
\expandafter\def\csname PY@tok@gp\endcsname{\let\PY@bf=\textbf\def\PY@tc##1{\textcolor[rgb]{0.00,0.00,0.50}{##1}}}
\expandafter\def\csname PY@tok@gs\endcsname{\let\PY@bf=\textbf}
\expandafter\def\csname PY@tok@err\endcsname{\def\PY@bc##1{\setlength{\fboxsep}{0pt}\fcolorbox[rgb]{1.00,0.00,0.00}{1,1,1}{\strut ##1}}}
\expandafter\def\csname PY@tok@gt\endcsname{\def\PY@tc##1{\textcolor[rgb]{0.00,0.27,0.87}{##1}}}
\expandafter\def\csname PY@tok@mf\endcsname{\def\PY@tc##1{\textcolor[rgb]{0.40,0.40,0.40}{##1}}}
\expandafter\def\csname PY@tok@nd\endcsname{\def\PY@tc##1{\textcolor[rgb]{0.67,0.13,1.00}{##1}}}
\expandafter\def\csname PY@tok@si\endcsname{\let\PY@bf=\textbf\def\PY@tc##1{\textcolor[rgb]{0.73,0.40,0.53}{##1}}}
\expandafter\def\csname PY@tok@s2\endcsname{\def\PY@tc##1{\textcolor[rgb]{0.73,0.13,0.13}{##1}}}
\expandafter\def\csname PY@tok@vm\endcsname{\def\PY@tc##1{\textcolor[rgb]{0.10,0.09,0.49}{##1}}}
\expandafter\def\csname PY@tok@sc\endcsname{\def\PY@tc##1{\textcolor[rgb]{0.73,0.13,0.13}{##1}}}
\expandafter\def\csname PY@tok@kt\endcsname{\def\PY@tc##1{\textcolor[rgb]{0.69,0.00,0.25}{##1}}}
\expandafter\def\csname PY@tok@dl\endcsname{\def\PY@tc##1{\textcolor[rgb]{0.73,0.13,0.13}{##1}}}
\expandafter\def\csname PY@tok@go\endcsname{\def\PY@tc##1{\textcolor[rgb]{0.53,0.53,0.53}{##1}}}
\expandafter\def\csname PY@tok@vi\endcsname{\def\PY@tc##1{\textcolor[rgb]{0.10,0.09,0.49}{##1}}}
\expandafter\def\csname PY@tok@sd\endcsname{\let\PY@it=\textit\def\PY@tc##1{\textcolor[rgb]{0.73,0.13,0.13}{##1}}}
\expandafter\def\csname PY@tok@mo\endcsname{\def\PY@tc##1{\textcolor[rgb]{0.40,0.40,0.40}{##1}}}
\expandafter\def\csname PY@tok@kp\endcsname{\def\PY@tc##1{\textcolor[rgb]{0.00,0.50,0.00}{##1}}}
\expandafter\def\csname PY@tok@bp\endcsname{\def\PY@tc##1{\textcolor[rgb]{0.00,0.50,0.00}{##1}}}
\expandafter\def\csname PY@tok@nf\endcsname{\def\PY@tc##1{\textcolor[rgb]{0.00,0.00,1.00}{##1}}}
\expandafter\def\csname PY@tok@s1\endcsname{\def\PY@tc##1{\textcolor[rgb]{0.73,0.13,0.13}{##1}}}
\expandafter\def\csname PY@tok@c1\endcsname{\let\PY@it=\textit\def\PY@tc##1{\textcolor[rgb]{0.25,0.50,0.50}{##1}}}
\expandafter\def\csname PY@tok@il\endcsname{\def\PY@tc##1{\textcolor[rgb]{0.40,0.40,0.40}{##1}}}
\expandafter\def\csname PY@tok@nv\endcsname{\def\PY@tc##1{\textcolor[rgb]{0.10,0.09,0.49}{##1}}}
\expandafter\def\csname PY@tok@cm\endcsname{\let\PY@it=\textit\def\PY@tc##1{\textcolor[rgb]{0.25,0.50,0.50}{##1}}}
\expandafter\def\csname PY@tok@sb\endcsname{\def\PY@tc##1{\textcolor[rgb]{0.73,0.13,0.13}{##1}}}
\expandafter\def\csname PY@tok@s\endcsname{\def\PY@tc##1{\textcolor[rgb]{0.73,0.13,0.13}{##1}}}
\expandafter\def\csname PY@tok@vc\endcsname{\def\PY@tc##1{\textcolor[rgb]{0.10,0.09,0.49}{##1}}}
\expandafter\def\csname PY@tok@gu\endcsname{\let\PY@bf=\textbf\def\PY@tc##1{\textcolor[rgb]{0.50,0.00,0.50}{##1}}}
\expandafter\def\csname PY@tok@ch\endcsname{\let\PY@it=\textit\def\PY@tc##1{\textcolor[rgb]{0.25,0.50,0.50}{##1}}}
\expandafter\def\csname PY@tok@mb\endcsname{\def\PY@tc##1{\textcolor[rgb]{0.40,0.40,0.40}{##1}}}

\def\PYZbs{\char`\\}
\def\PYZus{\char`\_}
\def\PYZob{\char`\{}
\def\PYZcb{\char`\}}
\def\PYZca{\char`\^}
\def\PYZam{\char`\&}
\def\PYZlt{\char`\<}
\def\PYZgt{\char`\>}
\def\PYZsh{\char`\#}
\def\PYZpc{\char`\%}
\def\PYZdl{\char`\$}
\def\PYZhy{\char`\-}
\def\PYZsq{\char`\'}
\def\PYZdq{\char`\"}
\def\PYZti{\char`\~}
% for compatibility with earlier versions
\def\PYZat{@}
\def\PYZlb{[}
\def\PYZrb{]}
\makeatother


    % Exact colors from NB
    \definecolor{incolor}{rgb}{0.0, 0.0, 0.5}
    \definecolor{outcolor}{rgb}{0.545, 0.0, 0.0}



    
    % Prevent overflowing lines due to hard-to-break entities
    \sloppy 
    % Setup hyperref package
    \hypersetup{
      breaklinks=true,  % so long urls are correctly broken across lines
      colorlinks=true,
      urlcolor=urlcolor,
      linkcolor=linkcolor,
      citecolor=citecolor,
      }
    % Slightly bigger margins than the latex defaults
    
    \geometry{verbose,tmargin=1in,bmargin=1in,lmargin=1in,rmargin=1in}
    
    

    \begin{document}
    
    
    \maketitle
    
    

    
    \hypertarget{self-driving-car-engineer-nanodegree}{%
\section{Self-Driving Car Engineer
Nanodegree}\label{self-driving-car-engineer-nanodegree}}

\hypertarget{project-finding-lane-lines-on-the-road}{%
\subsection{\texorpdfstring{Project: \textbf{Finding Lane Lines on the
Road}}{Project: Finding Lane Lines on the Road}}\label{project-finding-lane-lines-on-the-road}}

\begin{center}\rule{0.5\linewidth}{\linethickness}\end{center}

The project has been implemented to identify lane lines on the road. A
pipeline processing is applied to a series of individual images and
ultimately a video stream. All the images from ``test\_images'' are read
and and processed. All videos from ``test\_videos'' directory are read
and processed. Ultimately, one line is drawn for the left side of the
lane, and one for the right hand side. The corresponding outputs are
written to test\_images\_output and test\_videos\_output directory.

\begin{center}\rule{0.5\linewidth}{\linethickness}\end{center}

    \hypertarget{imports-needed-for-the-project}{%
\subsubsection{Imports needed for the
project}\label{imports-needed-for-the-project}}

    \begin{Verbatim}[commandchars=\\\{\}]
{\color{incolor}In [{\color{incolor} }]:} \PY{k+kn}{import} \PY{n+nn}{matplotlib}\PY{n+nn}{.}\PY{n+nn}{pyplot} \PY{k}{as} \PY{n+nn}{plt}
        \PY{k+kn}{import} \PY{n+nn}{matplotlib}\PY{n+nn}{.}\PY{n+nn}{image} \PY{k}{as} \PY{n+nn}{mpimg}
        \PY{k+kn}{import} \PY{n+nn}{numpy} \PY{k}{as} \PY{n+nn}{np}
        \PY{k+kn}{import} \PY{n+nn}{cv2}
        \PY{o}{\PYZpc{}}\PY{k}{matplotlib} inline
        \PY{k+kn}{import} \PY{n+nn}{math}
        \PY{k+kn}{import} \PY{n+nn}{os}
        \PY{k+kn}{from} \PY{n+nn}{moviepy}\PY{n+nn}{.}\PY{n+nn}{editor} \PY{k}{import} \PY{n}{VideoFileClip}
        \PY{k+kn}{from} \PY{n+nn}{IPython}\PY{n+nn}{.}\PY{n+nn}{display} \PY{k}{import} \PY{n}{HTML}
\end{Verbatim}


    \hypertarget{the-different-util-functions-used-in-the-pipeline}{%
\subsubsection{The different util functions used in the
pipeline}\label{the-different-util-functions-used-in-the-pipeline}}

    \begin{Verbatim}[commandchars=\\\{\}]
{\color{incolor}In [{\color{incolor}149}]:} \PY{k}{def} \PY{n+nf}{grayscale}\PY{p}{(}\PY{n}{img}\PY{p}{)}\PY{p}{:}
              \PY{l+s+sd}{\PYZdq{}\PYZdq{}\PYZdq{}Applies the Grayscale transform}
          \PY{l+s+sd}{    This will return an image with only one color channel}
          \PY{l+s+sd}{    but NOTE: to see the returned image as grayscale}
          \PY{l+s+sd}{    (assuming your grayscaled image is called \PYZsq{}gray\PYZsq{})}
          \PY{l+s+sd}{    you should call plt.imshow(gray, cmap=\PYZsq{}gray\PYZsq{})\PYZdq{}\PYZdq{}\PYZdq{}}
              \PY{k}{return} \PY{n}{cv2}\PY{o}{.}\PY{n}{cvtColor}\PY{p}{(}\PY{n}{img}\PY{p}{,} \PY{n}{cv2}\PY{o}{.}\PY{n}{COLOR\PYZus{}RGB2GRAY}\PY{p}{)}
              \PY{c+c1}{\PYZsh{} Or use BGR2GRAY if you read an image with cv2.imread()}
              \PY{c+c1}{\PYZsh{} return cv2.cvtColor(img, cv2.COLOR\PYZus{}BGR2GRAY)}
              
          \PY{k}{def} \PY{n+nf}{canny}\PY{p}{(}\PY{n}{img}\PY{p}{,} \PY{n}{low\PYZus{}threshold}\PY{p}{,} \PY{n}{high\PYZus{}threshold}\PY{p}{)}\PY{p}{:}
              \PY{l+s+sd}{\PYZdq{}\PYZdq{}\PYZdq{}Applies the Canny transform\PYZdq{}\PYZdq{}\PYZdq{}}
              \PY{k}{return} \PY{n}{cv2}\PY{o}{.}\PY{n}{Canny}\PY{p}{(}\PY{n}{img}\PY{p}{,} \PY{n}{low\PYZus{}threshold}\PY{p}{,} \PY{n}{high\PYZus{}threshold}\PY{p}{)}
          
          \PY{k}{def} \PY{n+nf}{gaussian\PYZus{}blur}\PY{p}{(}\PY{n}{img}\PY{p}{,} \PY{n}{kernel\PYZus{}size}\PY{p}{)}\PY{p}{:}
              \PY{l+s+sd}{\PYZdq{}\PYZdq{}\PYZdq{}Applies a Gaussian Noise kernel\PYZdq{}\PYZdq{}\PYZdq{}}
              \PY{k}{return} \PY{n}{cv2}\PY{o}{.}\PY{n}{GaussianBlur}\PY{p}{(}\PY{n}{img}\PY{p}{,} \PY{p}{(}\PY{n}{kernel\PYZus{}size}\PY{p}{,} \PY{n}{kernel\PYZus{}size}\PY{p}{)}\PY{p}{,} \PY{l+m+mi}{0}\PY{p}{)}
          
          \PY{k}{def} \PY{n+nf}{region\PYZus{}of\PYZus{}interest}\PY{p}{(}\PY{n}{img}\PY{p}{,} \PY{n}{vertices}\PY{p}{)}\PY{p}{:}
              \PY{l+s+sd}{\PYZdq{}\PYZdq{}\PYZdq{}}
          \PY{l+s+sd}{    Applies an image mask.}
          \PY{l+s+sd}{    }
          \PY{l+s+sd}{    Only keeps the region of the image defined by the polygon}
          \PY{l+s+sd}{    formed from `vertices`. The rest of the image is set to black.}
          \PY{l+s+sd}{    \PYZdq{}\PYZdq{}\PYZdq{}}
              \PY{c+c1}{\PYZsh{}defining a blank mask to start with}
              \PY{n}{mask} \PY{o}{=} \PY{n}{np}\PY{o}{.}\PY{n}{zeros\PYZus{}like}\PY{p}{(}\PY{n}{img}\PY{p}{)}   
              
              \PY{c+c1}{\PYZsh{}defining a 3 channel or 1 channel color to fill the mask with depending on the input image}
              \PY{k}{if} \PY{n+nb}{len}\PY{p}{(}\PY{n}{img}\PY{o}{.}\PY{n}{shape}\PY{p}{)} \PY{o}{\PYZgt{}} \PY{l+m+mi}{2}\PY{p}{:}
                  \PY{n}{channel\PYZus{}count} \PY{o}{=} \PY{n}{img}\PY{o}{.}\PY{n}{shape}\PY{p}{[}\PY{l+m+mi}{2}\PY{p}{]}  \PY{c+c1}{\PYZsh{} i.e. 3 or 4 depending on your image}
                  \PY{n}{ignore\PYZus{}mask\PYZus{}color} \PY{o}{=} \PY{p}{(}\PY{l+m+mi}{255}\PY{p}{,}\PY{p}{)} \PY{o}{*} \PY{n}{channel\PYZus{}count}
              \PY{k}{else}\PY{p}{:}
                  \PY{n}{ignore\PYZus{}mask\PYZus{}color} \PY{o}{=} \PY{l+m+mi}{255}
                  
              \PY{c+c1}{\PYZsh{}filling pixels inside the polygon defined by \PYZdq{}vertices\PYZdq{} with the fill color    }
              \PY{n}{cv2}\PY{o}{.}\PY{n}{fillPoly}\PY{p}{(}\PY{n}{mask}\PY{p}{,} \PY{n}{vertices}\PY{p}{,} \PY{n}{ignore\PYZus{}mask\PYZus{}color}\PY{p}{)}
              
              \PY{c+c1}{\PYZsh{}returning the image only where mask pixels are nonzero}
              \PY{n}{masked\PYZus{}image} \PY{o}{=} \PY{n}{cv2}\PY{o}{.}\PY{n}{bitwise\PYZus{}and}\PY{p}{(}\PY{n}{img}\PY{p}{,} \PY{n}{mask}\PY{p}{)}
              \PY{k}{return} \PY{n}{masked\PYZus{}image}
          
          
          \PY{k}{def} \PY{n+nf}{draw\PYZus{}lines}\PY{p}{(}\PY{n}{img}\PY{p}{,} \PY{n}{lines}\PY{p}{,} \PY{n}{color}\PY{o}{=}\PY{p}{[}\PY{l+m+mi}{255}\PY{p}{,} \PY{l+m+mi}{0}\PY{p}{,} \PY{l+m+mi}{0}\PY{p}{]}\PY{p}{,} \PY{n}{thickness}\PY{o}{=}\PY{l+m+mi}{10}\PY{p}{)}\PY{p}{:}
              \PY{l+s+sd}{\PYZdq{}\PYZdq{}\PYZdq{}}
          \PY{l+s+sd}{    NOTE: this is the function you might want to use as a starting point once you want to}
          \PY{l+s+sd}{    average/extrapolate the line segments you detect to map out the full}
          \PY{l+s+sd}{    extent of the lane (going from the result shown in raw\PYZhy{}lines\PYZhy{}example.mp4}
          \PY{l+s+sd}{    to that shown in P1\PYZus{}example.mp4).}
          
          \PY{l+s+sd}{    Think about things like separating line segments by their}
          \PY{l+s+sd}{    slope ((y2\PYZhy{}y1)/(x2\PYZhy{}x1)) to decide which segments are part of the left}
          \PY{l+s+sd}{    line vs. the right line.  Then, you can average the position of each of}
          \PY{l+s+sd}{    the lines and extrapolate to the top and bottom of the lane.}
          
          \PY{l+s+sd}{    This function draws `lines` with `color` and `thickness`.}
          \PY{l+s+sd}{    Lines are drawn on the image inplace (mutates the image).}
          \PY{l+s+sd}{    If you want to make the lines semi\PYZhy{}transparent, think about combining}
          \PY{l+s+sd}{    this function with the weighted\PYZus{}img() function below}
          \PY{l+s+sd}{    \PYZdq{}\PYZdq{}\PYZdq{}}
              \PY{n}{right\PYZus{}lines} \PY{o}{=} \PY{p}{[}\PY{p}{]}
              \PY{n}{left\PYZus{}lines} \PY{o}{=} \PY{p}{[}\PY{p}{]}
              \PY{n}{m\PYZus{}threshold} \PY{o}{=} \PY{l+m+mf}{0.5}
             
              \PY{n}{right\PYZus{}lines}\PY{p}{,}\PY{n}{left\PYZus{}lines}\PY{o}{=}\PY{n}{find\PYZus{}left\PYZus{}and\PYZus{}right\PYZus{}lines}\PY{p}{(}\PY{n}{lines}\PY{p}{,}\PY{n}{m\PYZus{}threshold}\PY{p}{)}
              \PY{n}{right\PYZus{}x\PYZus{}value}\PY{p}{,} \PY{n}{corresponding\PYZus{}right\PYZus{}line} \PY{o}{=} \PY{n}{get\PYZus{}1Dlines}\PY{p}{(}\PY{n}{right\PYZus{}lines}\PY{p}{)}
              \PY{n}{left\PYZus{}x\PYZus{}value}\PY{p}{,} \PY{n}{corresponding\PYZus{}left\PYZus{}line} \PY{o}{=} \PY{n}{get\PYZus{}1Dlines}\PY{p}{(}\PY{n}{left\PYZus{}lines}\PY{p}{)}
              \PY{n}{width} \PY{o}{=} \PY{n}{img}\PY{o}{.}\PY{n}{shape}\PY{p}{[}\PY{l+m+mi}{1}\PY{p}{]} 
          
              \PY{c+c1}{\PYZsh{}  For right line the start point would be the least value from all right points}
              \PY{k}{if}\PY{p}{(}\PY{n+nb}{len}\PY{p}{(}\PY{n}{right\PYZus{}x\PYZus{}value}\PY{p}{)}\PY{o}{!=}\PY{l+m+mi}{0}\PY{p}{)}\PY{p}{:}
                  \PY{k}{if} \PY{n}{DEBUG\PYZus{}MODE}\PY{p}{:}
                      \PY{n+nb}{print}\PY{p}{(}\PY{l+s+s2}{\PYZdq{}}\PY{l+s+s2}{x1 \PYZhy{} }\PY{l+s+s2}{\PYZdq{}}\PY{o}{+}\PY{n+nb}{str}\PY{p}{(}\PY{n+nb}{int}\PY{p}{(}\PY{n+nb}{min}\PY{p}{(}\PY{n}{right\PYZus{}x\PYZus{}value}\PY{p}{)}\PY{p}{)}\PY{p}{)}\PY{o}{+}\PY{l+s+s2}{\PYZdq{}}\PY{l+s+s2}{ y1 \PYZhy{} }\PY{l+s+s2}{\PYZdq{}}\PY{o}{+}\PY{n+nb}{str}\PY{p}{(}\PY{n+nb}{int}\PY{p}{(}\PY{n}{corresponding\PYZus{}right\PYZus{}line}\PY{p}{(}\PY{n+nb}{int}\PY{p}{(}\PY{n+nb}{min}\PY{p}{(}\PY{n}{right\PYZus{}x\PYZus{}value}\PY{p}{)}\PY{p}{)}\PY{p}{)}\PY{p}{)}\PY{p}{)}\PY{p}{)}
                      \PY{n+nb}{print}\PY{p}{(}\PY{l+s+s2}{\PYZdq{}}\PY{l+s+s2}{x2 \PYZhy{} }\PY{l+s+s2}{\PYZdq{}}\PY{o}{+}\PY{n+nb}{str}\PY{p}{(}\PY{n}{width}\PY{p}{)}\PY{o}{+}\PY{l+s+s2}{\PYZdq{}}\PY{l+s+s2}{ y2 \PYZhy{} }\PY{l+s+s2}{\PYZdq{}}\PY{o}{+}\PY{n+nb}{str}\PY{p}{(}\PY{n+nb}{int}\PY{p}{(}\PY{n}{corresponding\PYZus{}right\PYZus{}line}\PY{p}{(}\PY{n}{width}\PY{p}{)}\PY{p}{)}\PY{p}{)}\PY{p}{)}
                  \PY{n}{start\PYZus{}x} \PY{o}{=} \PY{n+nb}{int}\PY{p}{(}\PY{n+nb}{min}\PY{p}{(}\PY{n}{right\PYZus{}x\PYZus{}value}\PY{p}{)}\PY{p}{)}
                  \PY{n}{cv2}\PY{o}{.}\PY{n}{line}\PY{p}{(}\PY{n}{img}\PY{p}{,}\PY{p}{(}\PY{n}{start\PYZus{}x}\PY{p}{,} \PY{n+nb}{int}\PY{p}{(}\PY{n}{corresponding\PYZus{}right\PYZus{}line}\PY{p}{(}\PY{n}{start\PYZus{}x}\PY{p}{)}\PY{p}{)}\PY{p}{)}\PY{p}{,}\PY{p}{(}\PY{n}{width}\PY{p}{,} \PY{n+nb}{int}\PY{p}{(}\PY{n}{corresponding\PYZus{}right\PYZus{}line}\PY{p}{(}\PY{n}{width}\PY{p}{)}\PY{p}{)}\PY{p}{)}\PY{p}{,} \PY{n}{color}\PY{p}{,} \PY{n}{thickness}\PY{p}{)}
          
              \PY{c+c1}{\PYZsh{}  For left line the start point would be 0 and end would be max of all left points}
              \PY{k}{if} \PY{n}{DEBUG\PYZus{}MODE}\PY{p}{:}
                  \PY{n+nb}{print}\PY{p}{(}\PY{l+s+s2}{\PYZdq{}}\PY{l+s+s2}{x1 \PYZhy{} }\PY{l+s+s2}{\PYZdq{}}\PY{o}{+}\PY{n+nb}{str}\PY{p}{(}\PY{l+m+mi}{0}\PY{p}{)}\PY{o}{+}\PY{l+s+s2}{\PYZdq{}}\PY{l+s+s2}{ y1 \PYZhy{} }\PY{l+s+s2}{\PYZdq{}}\PY{o}{+}\PY{n+nb}{str}\PY{p}{(}\PY{n+nb}{int}\PY{p}{(}\PY{n}{corresponding\PYZus{}left\PYZus{}line}\PY{p}{(}\PY{l+m+mi}{0}\PY{p}{)}\PY{p}{)}\PY{p}{)}\PY{p}{)}
                  \PY{n+nb}{print}\PY{p}{(}\PY{l+s+s2}{\PYZdq{}}\PY{l+s+s2}{x2 \PYZhy{} }\PY{l+s+s2}{\PYZdq{}}\PY{o}{+}\PY{n+nb}{str}\PY{p}{(}\PY{n+nb}{max}\PY{p}{(}\PY{n}{left\PYZus{}x\PYZus{}value}\PY{p}{)}\PY{p}{)}\PY{o}{+}\PY{l+s+s2}{\PYZdq{}}\PY{l+s+s2}{ y2 \PYZhy{} }\PY{l+s+s2}{\PYZdq{}}\PY{o}{+}\PY{n+nb}{str}\PY{p}{(}\PY{n+nb}{int}\PY{p}{(}\PY{n}{corresponding\PYZus{}left\PYZus{}line}\PY{p}{(}\PY{n+nb}{max}\PY{p}{(}\PY{n}{left\PYZus{}x\PYZus{}value}\PY{p}{)}\PY{p}{)}\PY{p}{)}\PY{p}{)}\PY{p}{)}
              \PY{n}{cv2}\PY{o}{.}\PY{n}{line}\PY{p}{(}\PY{n}{img}\PY{p}{,}\PY{p}{(}\PY{l+m+mi}{0}\PY{p}{,} \PY{n+nb}{int}\PY{p}{(}\PY{n}{corresponding\PYZus{}left\PYZus{}line}\PY{p}{(}\PY{l+m+mi}{0}\PY{p}{)}\PY{p}{)}\PY{p}{)}\PY{p}{,}\PY{p}{(}\PY{n+nb}{max}\PY{p}{(}\PY{n}{left\PYZus{}x\PYZus{}value}\PY{p}{)}\PY{p}{,} \PY{n+nb}{int}\PY{p}{(}\PY{n}{corresponding\PYZus{}left\PYZus{}line}\PY{p}{(}\PY{n+nb}{max}\PY{p}{(}\PY{n}{left\PYZus{}x\PYZus{}value}\PY{p}{)}\PY{p}{)}\PY{p}{)}\PY{p}{)}\PY{p}{,}\PY{n}{color}\PY{p}{,} \PY{n}{thickness}\PY{p}{)}
          
          
          \PY{k}{def} \PY{n+nf}{get\PYZus{}1Dlines}\PY{p}{(}\PY{n}{lines}\PY{p}{)}\PY{p}{:}
              \PY{l+s+sd}{\PYZsq{}\PYZsq{}\PYZsq{}\PYZsq{}}
          \PY{l+s+sd}{    From the start and points slope and constant of the line that fits the two points is calculated}
          \PY{l+s+sd}{    Result: x, slope and constant}
          \PY{l+s+sd}{    \PYZsq{}\PYZsq{}\PYZsq{}}
              \PY{n}{x} \PY{o}{=} \PY{p}{[}\PY{p}{]}
              \PY{n}{y} \PY{o}{=} \PY{p}{[}\PY{p}{]}
              \PY{k}{if}\PY{p}{(}\PY{n+nb}{len}\PY{p}{(}\PY{n}{lines}\PY{p}{)}\PY{o}{!=}\PY{l+m+mi}{0}\PY{p}{)}\PY{p}{:}
                  \PY{k}{for} \PY{n}{line} \PY{o+ow}{in} \PY{n}{lines}\PY{p}{:}
                      \PY{k}{for} \PY{n}{x1}\PY{p}{,} \PY{n}{y1}\PY{p}{,} \PY{n}{x2}\PY{p}{,} \PY{n}{y2} \PY{o+ow}{in} \PY{n}{line}\PY{p}{:}
                          \PY{n}{x} \PY{o}{+}\PY{o}{=} \PY{p}{[}\PY{n}{x1}\PY{p}{,} \PY{n}{x2}\PY{p}{]}
                          \PY{n}{y} \PY{o}{+}\PY{o}{=} \PY{p}{[}\PY{n}{y1}\PY{p}{,} \PY{n}{y2}\PY{p}{]}
                      \PY{c+c1}{\PYZsh{}Returns m and b (i.e) coefficients of the line}
                      \PY{n}{z} \PY{o}{=} \PY{n}{np}\PY{o}{.}\PY{n}{polyfit}\PY{p}{(}\PY{n}{x}\PY{p}{,} \PY{n}{y}\PY{p}{,} \PY{l+m+mi}{1}\PY{p}{)}
                      \PY{k}{if} \PY{n}{DEBUG\PYZus{}MODE}\PY{p}{:}
                          \PY{n+nb}{print}\PY{p}{(}\PY{l+s+s2}{\PYZdq{}}\PY{l+s+s2}{z \PYZhy{} }\PY{l+s+s2}{\PYZdq{}}\PY{p}{,}\PY{n}{z}\PY{p}{)}
                  \PY{c+c1}{\PYZsh{}Returns a 1D polynomial equation object}
                  \PY{k}{return} \PY{n}{x}\PY{p}{,} \PY{n}{np}\PY{o}{.}\PY{n}{poly1d}\PY{p}{(}\PY{n}{z}\PY{p}{)}
              \PY{k}{else}\PY{p}{:}
                  \PY{k}{return} \PY{n}{x}\PY{p}{,}\PY{p}{(}\PY{l+m+mi}{0}\PY{p}{,}\PY{l+m+mi}{0}\PY{p}{)}
          
          
          \PY{k}{def} \PY{n+nf}{find\PYZus{}left\PYZus{}and\PYZus{}right\PYZus{}lines}\PY{p}{(}\PY{n}{lines}\PY{p}{,} \PY{n}{m\PYZus{}threshold}\PY{p}{)}\PY{p}{:}
              \PY{l+s+sd}{\PYZdq{}\PYZdq{}\PYZdq{}}
          \PY{l+s+sd}{    :argument line: Line points under considerations}
          \PY{l+s+sd}{    :return: right line point list and left line point list}
          \PY{l+s+sd}{    \PYZdq{}\PYZdq{}\PYZdq{}}
              \PY{c+c1}{\PYZsh{} Left line has positive slope}
              \PY{c+c1}{\PYZsh{} Right has negative slope}
              \PY{c+c1}{\PYZsh{} We discard lines which are either horizontal or away from Region of interest.}
              \PY{n}{right\PYZus{}lines\PYZus{}point\PYZus{}list} \PY{o}{=} \PY{p}{[}\PY{p}{]}
              \PY{n}{left\PYZus{}lines\PYZus{}point\PYZus{}list} \PY{o}{=} \PY{p}{[}\PY{p}{]}
              \PY{k}{for} \PY{n}{line} \PY{o+ow}{in} \PY{n}{lines}\PY{p}{:}
                  \PY{k}{for} \PY{n}{x1}\PY{p}{,} \PY{n}{y1}\PY{p}{,} \PY{n}{x2}\PY{p}{,} \PY{n}{y2} \PY{o+ow}{in} \PY{n}{line}\PY{p}{:}
                      \PY{c+c1}{\PYZsh{} Check if x1 = x2  to avoid the slope being equal to infinity}
                      \PY{k}{if}\PY{p}{(}\PY{n}{x2} \PY{o+ow}{is} \PY{o+ow}{not} \PY{n}{x1}\PY{p}{)}\PY{p}{:}
                          \PY{n}{slope}\PY{o}{=}\PY{p}{(}\PY{n}{y2}\PY{o}{\PYZhy{}}\PY{n}{y1}\PY{p}{)}\PY{o}{/}\PY{p}{(}\PY{n}{x2}\PY{o}{\PYZhy{}}\PY{n}{x1}\PY{p}{)}
                      \PY{k}{else}\PY{p}{:}
                          \PY{n}{slope}\PY{o}{=}\PY{p}{(}\PY{n}{y2}\PY{o}{\PYZhy{}}\PY{n}{y1}\PY{p}{)}
                      \PY{k}{if} \PY{p}{(}\PY{n+nb}{abs}\PY{p}{(}\PY{n}{slope}\PY{p}{)} \PY{o}{\PYZgt{}} \PY{n}{m\PYZus{}threshold}\PY{p}{)} \PY{o+ow}{and} \PY{p}{(}\PY{n+nb}{abs}\PY{p}{(}\PY{n}{y2}\PY{o}{\PYZhy{}}\PY{n}{y1}\PY{p}{)}\PY{o}{\PYZgt{}}\PY{l+m+mi}{25}\PY{p}{)}\PY{p}{:}
                          \PY{k}{if} \PY{n}{slope} \PY{o}{\PYZlt{}}\PY{o}{=} \PY{l+m+mi}{0}\PY{p}{:}
                              \PY{n}{left\PYZus{}lines\PYZus{}point\PYZus{}list}\PY{o}{.}\PY{n}{append}\PY{p}{(}\PY{n}{line}\PY{p}{)}
                          \PY{k}{else}\PY{p}{:}
                              \PY{n}{right\PYZus{}lines\PYZus{}point\PYZus{}list}\PY{o}{.}\PY{n}{append}\PY{p}{(}\PY{n}{line}\PY{p}{)}
              \PY{k}{return} \PY{n}{right\PYZus{}lines\PYZus{}point\PYZus{}list}\PY{p}{,} \PY{n}{left\PYZus{}lines\PYZus{}point\PYZus{}list}
          
          \PY{k}{def} \PY{n+nf}{hough\PYZus{}lines}\PY{p}{(}\PY{n}{img}\PY{p}{,} \PY{n}{rho}\PY{p}{,} \PY{n}{theta}\PY{p}{,} \PY{n}{threshold}\PY{p}{,} \PY{n}{min\PYZus{}line\PYZus{}len}\PY{p}{,} \PY{n}{max\PYZus{}line\PYZus{}gap}\PY{p}{)}\PY{p}{:}
              \PY{l+s+sd}{\PYZdq{}\PYZdq{}\PYZdq{}}
          \PY{l+s+sd}{    `img` should be the output of a Canny transform. }
          \PY{l+s+sd}{    Returns an image with hough lines drawn.}
          \PY{l+s+sd}{    \PYZdq{}\PYZdq{}\PYZdq{}}
              \PY{n}{lines} \PY{o}{=} \PY{n}{cv2}\PY{o}{.}\PY{n}{HoughLinesP}\PY{p}{(}\PY{n}{img}\PY{p}{,} \PY{n}{rho}\PY{p}{,} \PY{n}{theta}\PY{p}{,} \PY{n}{threshold}\PY{p}{,} \PY{n}{np}\PY{o}{.}\PY{n}{array}\PY{p}{(}\PY{p}{[}\PY{p}{]}\PY{p}{)}\PY{p}{,} \PY{n}{minLineLength}\PY{o}{=}\PY{n}{min\PYZus{}line\PYZus{}len}\PY{p}{,} \PY{n}{maxLineGap}\PY{o}{=}\PY{n}{max\PYZus{}line\PYZus{}gap}\PY{p}{)}
              \PY{n}{line\PYZus{}img} \PY{o}{=} \PY{n}{np}\PY{o}{.}\PY{n}{zeros}\PY{p}{(}\PY{p}{(}\PY{n}{img}\PY{o}{.}\PY{n}{shape}\PY{p}{[}\PY{l+m+mi}{0}\PY{p}{]}\PY{p}{,} \PY{n}{img}\PY{o}{.}\PY{n}{shape}\PY{p}{[}\PY{l+m+mi}{1}\PY{p}{]}\PY{p}{,} \PY{l+m+mi}{3}\PY{p}{)}\PY{p}{,} \PY{n}{dtype}\PY{o}{=}\PY{n}{np}\PY{o}{.}\PY{n}{uint8}\PY{p}{)}
              \PY{n}{draw\PYZus{}lines}\PY{p}{(}\PY{n}{line\PYZus{}img}\PY{p}{,} \PY{n}{lines}\PY{p}{)}
              \PY{k}{return} \PY{n}{line\PYZus{}img}
          
          \PY{k}{def} \PY{n+nf}{weighted\PYZus{}img}\PY{p}{(}\PY{n}{img}\PY{p}{,} \PY{n}{initial\PYZus{}img}\PY{p}{,} \PY{n}{α}\PY{o}{=}\PY{l+m+mf}{0.8}\PY{p}{,} \PY{n}{β}\PY{o}{=}\PY{l+m+mf}{1.}\PY{p}{,} \PY{n}{λ}\PY{o}{=}\PY{l+m+mf}{0.}\PY{p}{)}\PY{p}{:}
              \PY{l+s+sd}{\PYZdq{}\PYZdq{}\PYZdq{}}
          \PY{l+s+sd}{    `img` is the output of the hough\PYZus{}lines(), An image with lines drawn on it.}
          \PY{l+s+sd}{    Should be a blank image (all black) with lines drawn on it.}
          \PY{l+s+sd}{    }
          \PY{l+s+sd}{    `initial\PYZus{}img` should be the image before any processing.}
          \PY{l+s+sd}{    }
          \PY{l+s+sd}{    The result image is computed as follows:}
          \PY{l+s+sd}{    }
          \PY{l+s+sd}{    initial\PYZus{}img * α + img * β + λ}
          \PY{l+s+sd}{    NOTE: initial\PYZus{}img and img must be the same shape!}
          \PY{l+s+sd}{    \PYZdq{}\PYZdq{}\PYZdq{}}
              \PY{k}{return} \PY{n}{cv2}\PY{o}{.}\PY{n}{addWeighted}\PY{p}{(}\PY{n}{initial\PYZus{}img}\PY{p}{,} \PY{n}{α}\PY{p}{,} \PY{n}{img}\PY{p}{,} \PY{n}{β}\PY{p}{,} \PY{n}{λ}\PY{p}{)}
\end{Verbatim}


    \hypertarget{the-processing-pipeline}{%
\subsubsection{The Processing Pipeline}\label{the-processing-pipeline}}

    \begin{Verbatim}[commandchars=\\\{\}]
{\color{incolor}In [{\color{incolor}164}]:} \PY{k}{def} \PY{n+nf}{process\PYZus{}image}\PY{p}{(}\PY{n}{image\PYZus{}copy}\PY{p}{)}\PY{p}{:}
              \PY{c+c1}{\PYZsh{} NOTE: The output you return should be a color image (3 channel) for processing video below}
              \PY{c+c1}{\PYZsh{} TODO: put your pipeline here,}
              \PY{c+c1}{\PYZsh{} you should return the final output (image where lines are drawn on lanes)}
              
              \PY{c+c1}{\PYZsh{}Grayscaling the image to detect edges}
              \PY{n}{gray\PYZus{}image}\PY{o}{=}\PY{n}{grayscale}\PY{p}{(}\PY{n}{image\PYZus{}copy}\PY{p}{)}
              
              \PY{c+c1}{\PYZsh{}Applying an explicit Gaussian Blur of kernel size = 3}
              \PY{n}{gaussian\PYZus{}image}\PY{o}{=}\PY{n}{gaussian\PYZus{}blur}\PY{p}{(}\PY{n}{gray\PYZus{}image}\PY{p}{,}\PY{l+m+mi}{3}\PY{p}{)}
              
              \PY{c+c1}{\PYZsh{}Applyting Canny Edge Detection with custom tuned parameters}
              \PY{n}{low\PYZus{}threshold} \PY{o}{=} \PY{l+m+mi}{50}
              \PY{n}{high\PYZus{}threshold} \PY{o}{=} \PY{l+m+mi}{150}
              \PY{n}{canny\PYZus{}image}\PY{o}{=}\PY{n}{canny}\PY{p}{(}\PY{n}{gaussian\PYZus{}image}\PY{p}{,} \PY{n}{low\PYZus{}threshold}\PY{p}{,} \PY{n}{high\PYZus{}threshold}\PY{p}{)}
              \PY{n}{masked\PYZus{}image}\PY{o}{=}\PY{n}{region\PYZus{}of\PYZus{}interest}\PY{p}{(}\PY{n}{canny\PYZus{}image}\PY{p}{,}\PY{n}{mask\PYZus{}points}\PY{p}{)}
              
              \PY{c+c1}{\PYZsh{}Applying Hough Transform to detect line segments along with extrapolation}
              \PY{n}{rho} \PY{o}{=} \PY{l+m+mi}{2} \PY{c+c1}{\PYZsh{} distance resolution in pixels of the Hough grid}
              \PY{n}{theta} \PY{o}{=} \PY{n}{np}\PY{o}{.}\PY{n}{pi}\PY{o}{/}\PY{l+m+mi}{180} \PY{c+c1}{\PYZsh{} angular resolution in radians of the Hough grid}
              \PY{n}{threshold} \PY{o}{=} \PY{l+m+mi}{15}     \PY{c+c1}{\PYZsh{} minimum number of votes (intersections in Hough grid cell)}
              \PY{n}{min\PYZus{}line\PYZus{}length} \PY{o}{=} \PY{l+m+mi}{40} \PY{c+c1}{\PYZsh{}minimum number of pixels making up a line}
              \PY{n}{max\PYZus{}line\PYZus{}gap} \PY{o}{=} \PY{l+m+mi}{20}    \PY{c+c1}{\PYZsh{} maximum gap in pixels between connectable line segments}
              \PY{n}{hough\PYZus{}image}\PY{o}{=}\PY{n}{hough\PYZus{}lines}\PY{p}{(}\PY{n}{masked\PYZus{}image}\PY{p}{,} \PY{n}{rho}\PY{p}{,} \PY{n}{theta}\PY{p}{,} \PY{n}{threshold}\PY{p}{,} \PY{n}{min\PYZus{}line\PYZus{}length}\PY{p}{,} \PY{n}{max\PYZus{}line\PYZus{}gap}\PY{p}{)}
              \PY{c+c1}{\PYZsh{} Displaying all the intermediate outputs \PYZhy{} gray\PYZus{}image, canny\PYZus{}image, masked\PYZus{}image, hough\PYZus{}image}
              \PY{n}{f}\PY{p}{,} \PY{p}{(}\PY{p}{(}\PY{n}{ax1}\PY{p}{,} \PY{n}{ax2}\PY{p}{)}\PY{p}{,} \PY{p}{(}\PY{n}{ax3}\PY{p}{,} \PY{n}{ax4}\PY{p}{)}\PY{p}{)} \PY{o}{=} \PY{n}{plt}\PY{o}{.}\PY{n}{subplots}\PY{p}{(}\PY{l+m+mi}{2}\PY{p}{,} \PY{l+m+mi}{2}\PY{p}{,} \PY{n}{sharex}\PY{o}{=}\PY{l+s+s1}{\PYZsq{}}\PY{l+s+s1}{col}\PY{l+s+s1}{\PYZsq{}}\PY{p}{,} \PY{n}{sharey}\PY{o}{=}\PY{l+s+s1}{\PYZsq{}}\PY{l+s+s1}{row}\PY{l+s+s1}{\PYZsq{}}\PY{p}{)}
              \PY{n}{ax1}\PY{o}{.}\PY{n}{imshow}\PY{p}{(}\PY{n}{gray\PYZus{}image}\PY{p}{,}\PY{n}{cmap}\PY{o}{=}\PY{l+s+s1}{\PYZsq{}}\PY{l+s+s1}{gray}\PY{l+s+s1}{\PYZsq{}}\PY{p}{)}
              \PY{n}{ax1}\PY{o}{.}\PY{n}{set\PYZus{}title}\PY{p}{(}\PY{l+s+s1}{\PYZsq{}}\PY{l+s+s1}{Gray Scaled Image}\PY{l+s+s1}{\PYZsq{}}\PY{p}{)}
              \PY{n}{ax2}\PY{o}{.}\PY{n}{imshow}\PY{p}{(}\PY{n}{canny\PYZus{}image}\PY{p}{,}\PY{n}{cmap}\PY{o}{=}\PY{l+s+s1}{\PYZsq{}}\PY{l+s+s1}{gray}\PY{l+s+s1}{\PYZsq{}}\PY{p}{)}
              \PY{n}{ax2}\PY{o}{.}\PY{n}{set\PYZus{}title}\PY{p}{(}\PY{l+s+s1}{\PYZsq{}}\PY{l+s+s1}{Applied Canny Edge Detection}\PY{l+s+s1}{\PYZsq{}}\PY{p}{)}
              \PY{n}{ax3}\PY{o}{.}\PY{n}{imshow}\PY{p}{(}\PY{n}{masked\PYZus{}image}\PY{p}{,}\PY{n}{cmap}\PY{o}{=}\PY{l+s+s1}{\PYZsq{}}\PY{l+s+s1}{gray}\PY{l+s+s1}{\PYZsq{}}\PY{p}{)}
              \PY{n}{ax3}\PY{o}{.}\PY{n}{set\PYZus{}title}\PY{p}{(}\PY{l+s+s1}{\PYZsq{}}\PY{l+s+s1}{Applied ROI Mask}\PY{l+s+s1}{\PYZsq{}}\PY{p}{)}
              \PY{n}{ax4}\PY{o}{.}\PY{n}{imshow}\PY{p}{(}\PY{n}{hough\PYZus{}image}\PY{p}{)}
              \PY{n}{ax3}\PY{o}{.}\PY{n}{set\PYZus{}title}\PY{p}{(}\PY{l+s+s1}{\PYZsq{}}\PY{l+s+s1}{Hough Transform with Extrapolation}\PY{l+s+s1}{\PYZsq{}}\PY{p}{)}
              
              \PY{c+c1}{\PYZsh{} Merging the Lane Lines with the original image}
              \PY{n}{result} \PY{o}{=} \PY{n}{weighted\PYZus{}img}\PY{p}{(}\PY{n}{hough\PYZus{}image}\PY{p}{,} \PY{n}{image\PYZus{}copy}\PY{o}{.}\PY{n}{astype}\PY{p}{(}\PY{l+s+s1}{\PYZsq{}}\PY{l+s+s1}{uint8}\PY{l+s+s1}{\PYZsq{}}\PY{p}{)}\PY{p}{)}
              \PY{k}{return} \PY{n}{result}
\end{Verbatim}


    \hypertarget{lane-detection---images}{%
\subsubsection{Lane detection - Images}\label{lane-detection---images}}

Lane detection is applied to the videos present in the
\textbf{test\_images} folder and the output is written to
\textbf{test\_images\_output} directory

    \begin{Verbatim}[commandchars=\\\{\}]
{\color{incolor}In [{\color{incolor}166}]:} \PY{k}{if} \PY{n+nv+vm}{\PYZus{}\PYZus{}name\PYZus{}\PYZus{}} \PY{o}{==} \PY{l+s+s1}{\PYZsq{}}\PY{l+s+s1}{\PYZus{}\PYZus{}main\PYZus{}\PYZus{}}\PY{l+s+s1}{\PYZsq{}}\PY{p}{:}
              \PY{n}{DEBUG\PYZus{}MODE}\PY{o}{=}\PY{k+kc}{False}
              \PY{n}{test\PYZus{}image\PYZus{}list}\PY{o}{=}\PY{n}{os}\PY{o}{.}\PY{n}{listdir}\PY{p}{(}\PY{l+s+s2}{\PYZdq{}}\PY{l+s+s2}{test\PYZus{}images/}\PY{l+s+s2}{\PYZdq{}}\PY{p}{)}
              \PY{k}{for} \PY{n}{image\PYZus{}obj} \PY{o+ow}{in} \PY{n}{test\PYZus{}image\PYZus{}list}\PY{p}{:}
                  \PY{k}{if} \PY{n}{image\PYZus{}obj}\PY{o}{!=}\PY{l+s+s2}{\PYZdq{}}\PY{l+s+s2}{.DS\PYZus{}Store}\PY{l+s+s2}{\PYZdq{}}\PY{p}{:}
                      \PY{c+c1}{\PYZsh{}reading in an image}
                      \PY{n}{image} \PY{o}{=} \PY{n}{mpimg}\PY{o}{.}\PY{n}{imread}\PY{p}{(}\PY{l+s+s1}{\PYZsq{}}\PY{l+s+s1}{test\PYZus{}images/}\PY{l+s+s1}{\PYZsq{}}\PY{o}{+}\PY{n}{image\PYZus{}obj}\PY{p}{)}
                      \PY{c+c1}{\PYZsh{}printing out some stats and plotting}
                      \PY{n+nb}{print}\PY{p}{(}\PY{l+s+s1}{\PYZsq{}}\PY{l+s+s1}{Processing:}\PY{l+s+s1}{\PYZsq{}}\PY{p}{,} \PY{n}{image\PYZus{}obj}\PY{p}{,} \PY{l+s+s1}{\PYZsq{}}\PY{l+s+s1}{with dimensions:}\PY{l+s+s1}{\PYZsq{}}\PY{p}{,} \PY{n}{image}\PY{o}{.}\PY{n}{shape}\PY{p}{)}
                      \PY{n}{image\PYZus{}copy} \PY{o}{=} \PY{n}{image}\PY{o}{.}\PY{n}{copy}\PY{p}{(}\PY{p}{)}
                      \PY{n}{imshape} \PY{o}{=} \PY{n}{image\PYZus{}copy}\PY{o}{.}\PY{n}{shape}
                      \PY{n}{mask\PYZus{}points} \PY{o}{=} \PY{n}{np}\PY{o}{.}\PY{n}{array}\PY{p}{(}\PY{p}{[}\PY{p}{[}\PY{l+m+mi}{0}\PY{p}{,}\PY{n}{imshape}\PY{p}{[}\PY{l+m+mi}{0}\PY{p}{]}\PY{p}{]}\PY{p}{,}\PY{p}{[}\PY{l+m+mi}{470}\PY{p}{,} \PY{l+m+mi}{315}\PY{p}{]}\PY{p}{,} \PY{p}{[}\PY{l+m+mi}{490}\PY{p}{,} \PY{l+m+mi}{315}\PY{p}{]}\PY{p}{,} \PY{p}{[}\PY{n}{imshape}\PY{p}{[}\PY{l+m+mi}{1}\PY{p}{]}\PY{p}{,}\PY{n}{imshape}\PY{p}{[}\PY{l+m+mi}{0}\PY{p}{]}\PY{p}{]}\PY{p}{]}\PY{p}{)}
                      \PY{n}{mask\PYZus{}points}\PY{o}{=}\PY{n}{np}\PY{o}{.}\PY{n}{int32}\PY{p}{(}\PY{p}{[}\PY{n}{mask\PYZus{}points}\PY{p}{]}\PY{p}{)}
                      \PY{n}{result}\PY{o}{=}\PY{n}{process\PYZus{}image}\PY{p}{(}\PY{n}{image\PYZus{}copy}\PY{p}{)}
                      \PY{n}{mpimg}\PY{o}{.}\PY{n}{imsave}\PY{p}{(}\PY{l+s+s2}{\PYZdq{}}\PY{l+s+s2}{test\PYZus{}images\PYZus{}output/}\PY{l+s+s2}{\PYZdq{}}\PY{o}{+}\PY{n}{image\PYZus{}obj}\PY{p}{,} \PY{n}{result}\PY{p}{)}
                      \PY{n}{plt}\PY{o}{.}\PY{n}{figure}\PY{p}{(}\PY{p}{)}
                      \PY{n}{plt}\PY{o}{.}\PY{n}{imshow}\PY{p}{(}\PY{n}{result}\PY{p}{)}
                      \PY{n}{plt}\PY{o}{.}\PY{n}{title}\PY{p}{(}\PY{n}{image\PYZus{}obj}\PY{p}{)}
\end{Verbatim}


    \begin{Verbatim}[commandchars=\\\{\}]
Processing: solidYellowCurve.jpg with dimensions: (540, 960, 3)
Processing: solidYellowLeft.jpg with dimensions: (540, 960, 3)
Processing: solidYellowCurve2.jpg with dimensions: (540, 960, 3)
Processing: solidWhiteRight.jpg with dimensions: (540, 960, 3)
Processing: whiteCarLaneSwitch.jpg with dimensions: (540, 960, 3)
Processing: solidWhiteCurve.jpg with dimensions: (540, 960, 3)

    \end{Verbatim}

    \begin{center}
    \adjustimage{max size={0.9\linewidth}{0.9\paperheight}}{output_8_1.png}
    \end{center}
    { \hspace*{\fill} \\}
    
    \begin{center}
    \adjustimage{max size={0.9\linewidth}{0.9\paperheight}}{output_8_2.png}
    \end{center}
    { \hspace*{\fill} \\}
    
    \begin{center}
    \adjustimage{max size={0.9\linewidth}{0.9\paperheight}}{output_8_3.png}
    \end{center}
    { \hspace*{\fill} \\}
    
    \begin{center}
    \adjustimage{max size={0.9\linewidth}{0.9\paperheight}}{output_8_4.png}
    \end{center}
    { \hspace*{\fill} \\}
    
    \begin{center}
    \adjustimage{max size={0.9\linewidth}{0.9\paperheight}}{output_8_5.png}
    \end{center}
    { \hspace*{\fill} \\}
    
    \begin{center}
    \adjustimage{max size={0.9\linewidth}{0.9\paperheight}}{output_8_6.png}
    \end{center}
    { \hspace*{\fill} \\}
    
    \begin{center}
    \adjustimage{max size={0.9\linewidth}{0.9\paperheight}}{output_8_7.png}
    \end{center}
    { \hspace*{\fill} \\}
    
    \begin{center}
    \adjustimage{max size={0.9\linewidth}{0.9\paperheight}}{output_8_8.png}
    \end{center}
    { \hspace*{\fill} \\}
    
    \begin{center}
    \adjustimage{max size={0.9\linewidth}{0.9\paperheight}}{output_8_9.png}
    \end{center}
    { \hspace*{\fill} \\}
    
    \begin{center}
    \adjustimage{max size={0.9\linewidth}{0.9\paperheight}}{output_8_10.png}
    \end{center}
    { \hspace*{\fill} \\}
    
    \begin{center}
    \adjustimage{max size={0.9\linewidth}{0.9\paperheight}}{output_8_11.png}
    \end{center}
    { \hspace*{\fill} \\}
    
    \begin{center}
    \adjustimage{max size={0.9\linewidth}{0.9\paperheight}}{output_8_12.png}
    \end{center}
    { \hspace*{\fill} \\}
    
    \hypertarget{lane-detection---videos}{%
\subsubsection{Lane detection - Videos}\label{lane-detection---videos}}

Lane detection is applied to the videos present in the
\textbf{test\_video} folder and the output is written to
\textbf{test\_videos\_output} directory

    \begin{Verbatim}[commandchars=\\\{\}]
{\color{incolor}In [{\color{incolor} }]:} \PY{n}{test\PYZus{}videos\PYZus{}list}\PY{o}{=}\PY{n}{os}\PY{o}{.}\PY{n}{listdir}\PY{p}{(}\PY{l+s+s2}{\PYZdq{}}\PY{l+s+s2}{test\PYZus{}videos/}\PY{l+s+s2}{\PYZdq{}}\PY{p}{)}
        \PY{k}{for} \PY{n}{video\PYZus{}obj} \PY{o+ow}{in} \PY{n}{test\PYZus{}videos\PYZus{}list}\PY{p}{:}
            \PY{k}{if} \PY{n}{video\PYZus{}obj}\PY{o}{!=}\PY{l+s+s2}{\PYZdq{}}\PY{l+s+s2}{.DS\PYZus{}Store}\PY{l+s+s2}{\PYZdq{}}\PY{p}{:}
                \PY{n}{video\PYZus{}output} \PY{o}{=} \PY{l+s+s1}{\PYZsq{}}\PY{l+s+s1}{test\PYZus{}videos\PYZus{}output/}\PY{l+s+s1}{\PYZsq{}}\PY{o}{+}\PY{n}{video\PYZus{}obj}
                \PY{c+c1}{\PYZsh{}\PYZsh{} To speed up the testing process you may want to try your pipeline on a shorter subclip of the video}
                \PY{c+c1}{\PYZsh{}\PYZsh{} To do so add .subclip(start\PYZus{}second,end\PYZus{}second) to the end of the line below}
                \PY{c+c1}{\PYZsh{}\PYZsh{} Where start\PYZus{}second and end\PYZus{}second are integer values representing the start and end of the subclip}
                \PY{c+c1}{\PYZsh{}\PYZsh{} You may also uncomment the following line for a subclip of the first 5 seconds}
                \PY{c+c1}{\PYZsh{}\PYZsh{}clip1 = VideoFileClip(\PYZdq{}test\PYZus{}videos/solidWhiteRight.mp4\PYZdq{}).subclip(0,5)}
                \PY{n}{clip1} \PY{o}{=} \PY{n}{VideoFileClip}\PY{p}{(}\PY{l+s+s2}{\PYZdq{}}\PY{l+s+s2}{test\PYZus{}videos/}\PY{l+s+s2}{\PYZdq{}}\PY{o}{+}\PY{n}{video\PYZus{}obj}\PY{p}{)}\PY{o}{.}\PY{n}{subclip}\PY{p}{(}\PY{l+m+mi}{0}\PY{p}{,}\PY{l+m+mi}{5}\PY{p}{)}
                \PY{n}{white\PYZus{}clip} \PY{o}{=} \PY{n}{clip1}\PY{o}{.}\PY{n}{fl\PYZus{}image}\PY{p}{(}\PY{n}{process\PYZus{}image}\PY{p}{)} \PY{c+c1}{\PYZsh{}NOTE: this function expects color images!!}
                \PY{o}{\PYZpc{}}\PY{k}{time} white\PYZus{}clip.write\PYZus{}videofile(video\PYZus{}output, audio=False)
        
        \PY{c+c1}{\PYZsh{}Will Display the last video}
        \PY{n}{HTML}\PY{p}{(}\PY{l+s+s2}{\PYZdq{}\PYZdq{}\PYZdq{}}
        \PY{l+s+s2}{    \PYZlt{}video width=}\PY{l+s+s2}{\PYZdq{}}\PY{l+s+s2}{960}\PY{l+s+s2}{\PYZdq{}}\PY{l+s+s2}{ height=}\PY{l+s+s2}{\PYZdq{}}\PY{l+s+s2}{540}\PY{l+s+s2}{\PYZdq{}}\PY{l+s+s2}{ controls\PYZgt{}}
        \PY{l+s+s2}{      \PYZlt{}source src=}\PY{l+s+s2}{\PYZdq{}}\PY{l+s+si}{\PYZob{}0\PYZcb{}}\PY{l+s+s2}{\PYZdq{}}\PY{l+s+s2}{\PYZgt{}}
        \PY{l+s+s2}{    \PYZlt{}/video\PYZgt{}}
        \PY{l+s+s2}{    }\PY{l+s+s2}{\PYZdq{}\PYZdq{}\PYZdq{}}\PY{o}{.}\PY{n}{format}\PY{p}{(}\PY{n}{video\PYZus{}output}\PY{p}{)}\PY{p}{)}
\end{Verbatim}


    \begin{Verbatim}[commandchars=\\\{\}]
[MoviePy] >>>> Building video test\_videos\_output/solidWhiteRight.mp4
[MoviePy] Writing video test\_videos\_output/solidWhiteRight.mp4

    \end{Verbatim}

    \begin{Verbatim}[commandchars=\\\{\}]
 99\%|█████████▉| 125/126 [00:23<00:00,  5.79it/s]

    \end{Verbatim}

    \begin{Verbatim}[commandchars=\\\{\}]
[MoviePy] Done.
[MoviePy] >>>> Video ready: test\_videos\_output/solidWhiteRight.mp4 

CPU times: user 21.7 s, sys: 884 ms, total: 22.6 s
Wall time: 24.3 s
[MoviePy] >>>> Building video test\_videos\_output/solidYellowLeft.mp4
[MoviePy] Writing video test\_videos\_output/solidYellowLeft.mp4

    \end{Verbatim}

    \begin{Verbatim}[commandchars=\\\{\}]
 99\%|█████████▉| 125/126 [00:25<00:00,  5.02it/s]

    \end{Verbatim}

    \begin{Verbatim}[commandchars=\\\{\}]
[MoviePy] Done.
[MoviePy] >>>> Video ready: test\_videos\_output/solidYellowLeft.mp4 

CPU times: user 23.3 s, sys: 966 ms, total: 24.2 s
Wall time: 26.2 s

    \end{Verbatim}

\begin{Verbatim}[commandchars=\\\{\}]
{\color{outcolor}Out[{\color{outcolor} }]:} <IPython.core.display.HTML object>
\end{Verbatim}
            
    \begin{center}
    \adjustimage{max size={0.9\linewidth}{0.9\paperheight}}{output_10_6.png}
    \end{center}
    { \hspace*{\fill} \\}
    
    \begin{center}
    \adjustimage{max size={0.9\linewidth}{0.9\paperheight}}{output_10_7.png}
    \end{center}
    { \hspace*{\fill} \\}
    
    \begin{center}
    \adjustimage{max size={0.9\linewidth}{0.9\paperheight}}{output_10_8.png}
    \end{center}
    { \hspace*{\fill} \\}
    
    \begin{center}
    \adjustimage{max size={0.9\linewidth}{0.9\paperheight}}{output_10_9.png}
    \end{center}
    { \hspace*{\fill} \\}
    
    \begin{center}
    \adjustimage{max size={0.9\linewidth}{0.9\paperheight}}{output_10_10.png}
    \end{center}
    { \hspace*{\fill} \\}
    
    \begin{center}
    \adjustimage{max size={0.9\linewidth}{0.9\paperheight}}{output_10_11.png}
    \end{center}
    { \hspace*{\fill} \\}
    
    \begin{center}
    \adjustimage{max size={0.9\linewidth}{0.9\paperheight}}{output_10_12.png}
    \end{center}
    { \hspace*{\fill} \\}
    
    \begin{center}
    \adjustimage{max size={0.9\linewidth}{0.9\paperheight}}{output_10_13.png}
    \end{center}
    { \hspace*{\fill} \\}
    
    \begin{center}
    \adjustimage{max size={0.9\linewidth}{0.9\paperheight}}{output_10_14.png}
    \end{center}
    { \hspace*{\fill} \\}
    
    \begin{center}
    \adjustimage{max size={0.9\linewidth}{0.9\paperheight}}{output_10_15.png}
    \end{center}
    { \hspace*{\fill} \\}
    
    \hypertarget{conclusion}{%
\subsubsection{Conclusion}\label{conclusion}}

Thus, we successfully apply various image processing techniques in a
pipeline to mark the driving lanes in a color agnostic manner within the
region of interest.

    \hypertarget{future-improvements}{%
\subsubsection{Future Improvements}\label{future-improvements}}

\begin{itemize}
\tightlist
\item
  Be robust to image size
\item
  Be robust to image resolution
\item
  Be robust to the alignment of camera
\item
  Identify other markings within the lane
\end{itemize}


    % Add a bibliography block to the postdoc
    
    
    
    \end{document}
